\documentclass[12pt,a4paper]{article}
\usepackage[utf8]{inputenc}

\usepackage{subfiles}
\usepackage{lipsum} % lorem
\usepackage{scrextend} % addmargin
\usepackage[flushleft]{threeparttable} % ta
\usepackage{fontspec} % Arial
\usepackage[portuguese]{babel}
\usepackage{csquotes}
\usepackage[font={scriptsize}]{caption}
\usepackage{indentfirst} % indentação do primeiro parágrafo de seção
\usepackage{makeidx} % Títulos das tabelas
\usepackage[pagestyles]{titlesec} % titleformat
\usepackage{geometry}
\usepackage{amsfonts} % blackletter
\usepackage{graphicx} % resizebox
\usepackage{svg} % Grafos
\usepackage{enumitem} % HC01, HC02...
\usepackage{caption}
% citestyle authoryear para citações anbt
\usepackage[backend=biber,sorting=nty, style=abnt, citestyle=abnt-numeric]{biblatex}

\addto\captionsportuguese{
	\renewcommand{\contentsname}{ }
	\renewcommand{\contentsname}{ }
	\renewcommand{\listfigurename}{ }
	\renewcommand{\listtablename}{ }
}

\usepackage[breaklinks=true,
			pdfauthor={Léo Hardt},
			pdftitle={TCC - Léo Hardt},
			pdfsubject={Minerva:  Um Software para a Criação da Grade Horária de Escolas Brasileiras},
			pdfkeywords={Software, Escola, Grade Horária}]{hyperref}


\title{Minerva:  Um Software para a Criação da Grade Horária de Escolas Brasileiras}

\geometry{
	a4paper,
	left=30mm,
	right=20mm,
	top=30mm,
	bottom=20mm
}

\pagestyle{myheadings}

\addbibresource{texto.bib}

\setmainfont{Arial}
\setlength{\parindent}{12.5mm}
\renewcommand{\baselinestretch}{1.5}

\titleformat{\section}
  {\normalfont\bfseries}{\thesection}{1em}{}
\titleformat{\subsection}
	{\normalfont\bfseries}{\thesubsection}{1em}{}
\titleformat{\subsubsection}
	{\normalfont}{\thesubsubsection}{1em}{}

\newenvironment{bottompar}{\par\vspace*{\fill}}{\clearpage}

\begin{document}



	%%%%%%%%%%%%%%%%%%%%%%%%%%%%%%%%%%%
	%%%            Capa            %%%%
	%%%%%%%%%%%%%%%%%%%%%%%%%%%%%%%%%%%
	\thispagestyle{empty}

	\begin{center}
		INSTITUTO FEDERAL DE EDUCAÇÃO, CIÊNCIA E TECNOLOGIA DO RIO GRANDE DO SUL -- CAMPUS CANOAS \\
		CURSO TÉCNICO EM DESENVOLVIMENTO DE SISTEMAS INTEGRADO AO ENSINO MÉDIO\\
	\end{center}

	\vskip 3cm

	\begin{center}
		LÉO MARCO DE ASSIS HARDT
	\end{center}

	\vskip 5cm

	\begin{center}
		\textbf{Minerva:  Um Software para a Criação da Grade Horária de Escolas Brasileiras}
	\end{center}


	\begin{bottompar}
		\begin{center}
		CANOAS \\
		2020
		\end{center}
	\end{bottompar}

	%%%%%%%%%%%%%%%%%%%%%%%%%%%%%%%%%%%
	%%%       Folha de Rosto       %%%%
	%%%%%%%%%%%%%%%%%%%%%%%%%%%%%%%%%%%
	\thispagestyle{empty}

	\begin{center}
		LÉO MARCO DE ASSIS HARDT
	\end{center}

	\vskip 3cm


	\begin{center}
		\textbf{Minerva:  Um Software para a Criação da Grade Horária de Escolas Brasileiras}
	\end{center}

	\vskip 2cm

	\begin{addmargin}[7.5cm]{0em}

		\setlength{\parindent}{0mm}
		Trabalho de Conclusão de Curso apresentado como requisito parcial para obtenção do grau de Técnico em Desenvolvimento de Sistemas pelo Instituto Federal de Educação, Ciência e Tecnologia do Rio Grande do Sul – Campus Canoas.\\

		Prof. Gustavo Neuberger\\
		Orientador

	\end{addmargin}

	\begin{bottompar}
		\begin{center}
			CANOAS \\
			2020
		\end{center}
	\end{bottompar}


	%%%%%%%%%%%%%%%%%%%%%%%%%%%%%%%%%%%
	%%%       Agradecimentos       %%%%
	%%%%%%%%%%%%%%%%%%%%%%%%%%%%%%%%%%%
	\thispagestyle{empty}
	\section*{AGRADECIMENTOS}

	Gostaria de agradecer, em primeiro lugar, a todos que me apoiaram diretamente. Colegas, amigos, parentes e professores. Em segundo lugar, a todos aqueles que acreditaram num ensino de qualidade e disponibilizaram seu conhecimento ao mundo, mesmo que nunca se conheça o total impacto de tais ações. Nomes dos quais me lembro são Richard Feynmann, Grant Sanderson e Sal Khan. Por fim, gostaria de agradecer aos responsáveis pelo ambiente no qual tive a honra de me formar e no qual gostaria que muitos outros alunos tivessem a mesma oportunidade de evoução que eu tive.

	\newpage


	%%%%%%%%%%%%%%%%%%%%%%%%%%%%%%%%%%%
	%%%     Resumo em Português    %%%%
	%%%%%%%%%%%%%%%%%%%%%%%%%%%%%%%%%%%
	\thispagestyle{empty}
	\section*{RESUMO}

	No presente trabalho, é estudado o problema da elaboração do cronograma semanal escolar. São catalogadas necessidades e preferências de escolas em relação ao horário final, bem como carências de métodos atualmente utilizados. São consideradas soluções anteriores, sejam elas computacionais ou não. É identificada, então, uma necessidade das escolas brasileiras por um \textit{software} público e de fácil utilização para a elaboração do mesmo. Um software com esse fim é implementado utilizando uma abordagem de criação de horários semelhante à usada em motores de xadrez. Essa abordagem é comparada	computacional e utilitariamente com abordagens anteriores. O software final é testado e é feita uma pesquisa de satisfação com as escolas que participaram do teste. Os resultados são mostrados. São listados possíveis trabalhos futuros na área.


	\begingroup
		\setlength{\parindent}{0mm}
		\textbf{Palavras chave:} \textit{Software}; Escola; Grade Horária.
	\endgroup
	\newpage


	%%%%%%%%%%%%%%%%%%%%%%%%%%%%%%%%%%%
	%%%      Resumo em Inglês      %%%%
	%%%%%%%%%%%%%%%%%%%%%%%%%%%%%%%%%%%
	\thispagestyle{empty}
	\section*{ABSTRACT}

	In the present work, the school timetabling problem is studied. School necessities and preferences about the final timetable, as well as deficiencies of the current methods are listed. Previous solutions, be they computational or not, are considered. A necessity by the brazilian schools for an easy to use public timetabling software is identified. A software is implemented with this goal, using an approach similar to that used in chess engines. This approach is compared from computational and utilitarian perspectives. The software is tested, and a satisfaction survey is made for schools which participated in the test. The results are shown. Possible future work in the area is listed.

	\begingroup
		\setlength{\parindent}{0mm}
		\textbf{Keywords:} Software; School; Timetable.
	\endgroup

	\newpage


	% % É permitido:  https://www.puc-rio.br/ensinopesq/ccpg/normas/resumos_palavras_chave.html
	% % Mencionado aqui também: http://www.superclickmonografias.com/resumolinguaestrangeiramonografia.html
	% %%%%%%%%%%%%%%%%%%%%%%%%%%%%%%%%%%%
	% %%%      Resumo em Alemão      %%%%
	% %%%%%%%%%%%%%%%%%%%%%%%%%%%%%%%%%%%
	% \thispagestyle{empty}
	% \section*{ZUSAMMENFASSUNG}
	%
	% In der vorliegenden Arbeit, ist das Stundenplanungsproblem untergesucht. Schulnotwendigkeiten und Vorlieben, sowie Mängel der aktuellen Systeme sind aufgelistet. Vorherige Lösungen, rechnerisch oder nicht, sind betrachtet. Eine Notwendigkeit von brazilianer Schulen ist erkannt.
	%
	% \lipsum[1]
	%
	%
	% \begingroup
	% 	\setlength{\parindent}{0mm}
	% 	\textbf{Schlüsselwörter:} Wort1; Wort2; Wort3.
	% \endgroup
	%
	% \newpage

	%%%%%%%%%%%%%%%%%%%%%%%%%%%%%%%%%%%
	%%%      Lista de Tabelas      %%%%
	%%%%%%%%%%%%%%%%%%%%%%%%%%%%%%%%%%%
	\thispagestyle{empty}
	\section*{LISTA DE TABELAS}

	\listoftables

	\newpage

	%%%%%%%%%%%%%%%%%%%%%%%%%%%%%%%%%%%
	%%%      Lista de Figuras      %%%%
	%%%%%%%%%%%%%%%%%%%%%%%%%%%%%%%%%%%
	\thispagestyle{empty}
	\section*{LISTA DE FIGURAS}

	\listoffigures

	\newpage


	%%%%%%%%%%%%%%%%%%%%%%%%%%%%%%%%%%%
	%%%      Lista de Siglas       %%%%
	%%%%%%%%%%%%%%%%%%%%%%%%%%%%%%%%%%%
	\thispagestyle{empty}
	\section*{LISTA DE ABREVIATURAS E SIGLAS}

	\begin{tabular}{p{3cm} p{0.6\textwidth}}
	  C & \textit{C Programming Language} \\
	  IFRS & \textit{Instituto Federal de Educação, Ciência e Tecnologia do Rio Grande do Sul} \\
	  UML & \textit{Unified Modeling Language} \\
	  HC & \textit{Hard Constraint} \\
	  SC & \textit{Soft Constraint} \\
	  OC & \textit{Optional Constraint}
	\end{tabular}\\


	\newpage


	% %%%%%%%%%%%%%%%%%%%%%%%%%%%%%%%%%%%
	% %%%     Lista de Algorítmos    %%%%
	% %%%%%%%%%%%%%%%%%%%%%%%%%%%%%%%%%%%
	% \thispagestyle{empty}
	% \section*{LISTA DE ALGORÍTMOS}
	%
	% \listofalgorithms
	%
	% \newpage



	%%%%%%%%%%%%%%%%%%%%%%%%%%%%%%%%%%%
	%%%          Sumário           %%%%
	%%%%%%%%%%%%%%%%%%%%%%%%%%%%%%%%%%%
	\thispagestyle{empty}
	\section*{SUMÁRIO}

	\begingroup
		\let\clearpage\relax
		\vspace{-1cm} % Removed title space.
		\tableofcontents
	\endgroup

	\newpage


	%%%%%%%%%%%%%%%%%%%%%%%%%%%%%%%%%%%
	%%%         Introdução         %%%%
	%%%%%%%%%%%%%%%%%%%%%%%%%%%%%%%%%%%
	\section{INTRODUÇÃO}
		\subfile{sections/justificativa.tex}
	\newpage


	%%%%%%%%%%%%%%%%%%%%%%%%%%%%%%%%%%%
	%%%   Fundamentação Teórica    %%%%
	%%%%%%%%%%%%%%%%%%%%%%%%%%%%%%%%%%%
	\section{FUNDAMENTAÇÃO TEÓRICA}
		% \subsection{Descrição Formal do Cronograma}

		% \par Para a seguinte seção, pressupomos:
		%
		% \textbf{Pressuposições (ASs):}
		% \begin{enumerate}
		% 	\item O cronograma abrange um Ciclo, que é dividido em Dias e Períodos;
		% 	\item Há um número fixo de dias em um ciclo;
		% 	\item Há um número fixo de períodos em um dia, que é o mesmo para qualquer outro dia;
		% 	\item Todos os períodos tem igual tamanho;
		% 	\item O currículo de uma turma para cada disciplina é um número exato de períodos por ciclo;
		% \end{enumerate}

		% \par Podemos definir o cronograma da seguinte forma:
		%
		% \begin{itemize}
		% 	\item $P = \{p_1, p_2, p_3, ..., p_\pi\}$ é a lista de todos os $\pi$ períodos;
		% 	\item $T = \{t_1, t_2, t_3, ..., t_\theta\}$ é a lista de todos os $\theta$ professores;
		% 	\item $C = \{c_1, c_2, c_3, ..., c_\kappa\} $ é a lista de todas as $\kappa$ turmas;
		% 	\item $R = \{r_1, r_2, r_3, ..., r_\rho\}$ é a lista de todas as $\rho$ salas;
		% 	\item $L = \{l_1, l_2, l_3, ..., l_\lambda\}$ é a lista de todas as $\lambda$ aulas, sendo uma aula uma tupla $l = (p,t,c,r)$;
		% \end{itemize}

		% \par É chamada de cronograma a lista $L$ de todas as aulas em um ciclo.

		% \par Definiremos que um cronograma aloca dois tipos de eventos: aulas, envolvendo uma sala, um professor (ou grupo destes), uma turma (ou um grupo destas), um período e uma disciplina; e reuniões, envolvendo um grupo de professores e um período, podendo envolver uma sala, mas em geral esses eventos acontecem em uma chamada "sala dos professores". Repare que horários para a preparação de aulas podem ser
		%
		% \par Para elaborá-lo, temos conhecimento de todas as necessidades curriculares das turmas e somos encarregados de escolher um professor, uma sala e um período para cada uma dessas aulas. Essas escolhas devem levar em conta os seguintes fatores:
		%
		% \textbf{Requisitos Rígidos (HCs):}
		% \begin{enumerate}
		% 	\item Um professor pode comparecer a, no máximo, um evento por período;
		% 	\item Uma turma pode comparecer a, no máximo, um evento por período;
		% 	\item Uma sala pode ser ocupada por, no máximo, um evento por período;
		% 	\item O número de períodos utilizados para uma disciplina por uma turma em um ciclo deve ser o especificado no seu currículo;
		% 	\item Um único grupo de professores é responsável por ministrar as aulas de uma disciplina para uma turma, e deve se fazer presente em todas essas aulas;
		% 	\item Nenhuma sala será ocupada acima de sua capacidade;
		% 	\item Nenhuma aula será dada em uma sala inadequada para tal;
		% 	\item Uma aula só será ministrada por um professor ou grupo de professores que está capacitado para tal.
		% 	\item Um professor não ministrará aulas em uma sala inadequada para ele;
		% 	\item Uma turma não assistirá a aulas em uma sala inadequada para ela;
		% \end{enumerate}
		%
		% \par Também devem ser levadas em conta as seguintes preferências:
		%
		% \textbf{Requisitos Flexíveies (OCs):}
		% \begin{enumerate}
		% 	\item As preferências de cada turma, considerando o período em que a aula é dada;
		% 	\item As preferências de cada disciplina de cada turma, considerando o professor que dará a aula;
		% 	\item As preferências de cada disciplina de cada professor, considerando a geminação de períodos;
		% 	\item As preferências de cada sala considerando o período em que a aula é dada;
		% 	\item As preferências de cada professor, considerando a sala em que a aula é dada;
		% 	\item As preferências de cada aula, considerando o período em que a aula é dada;
		% \end{enumerate}
		%
		% \textbf{Requisitos Opcionais Rígidos (OCs):}
		% \begin{enumerate}
		% 	\item A escola pode escolher operar em um subconjunto dos períodos de um ciclo;
		% 	\item O professor poderá ter um conjunto de períodos para a elaboração de suas aulas;
		% 	\item Algumas aulas poderão ter salas, horários e professores previamente escolhidos;
		% 	\item Algumas turmas podem ter um número máximo de aulas por dia por grupos de disciplina;
		% 	\item Alguns professores podem ter um número máximo de aulas por dia por turma (mas sem discriminar qual turma);
		% 	\item Alguns professores podem ter um número máximo de dias para frequentar a instituição (discriminando o dia ou não);
		% 	\item Pode haver, para cada turma, um horário fixo de entrada e de saída da instituição;
		% 	\item Pode haver, para cada turma, a possibilidade de períodos vagos entre aulas;
		% 	\item Uma sala pode ter disponibilidade limitada durante um ciclo;
		% 	\item Uma turma pode ter disponibilidade limitada durante um ciclo;
		% 	\item Um professor pode ter disponibilidade limitada durante um ciclo;
		% \end{enumerate}

	\clearpage

	%%%%%%%%%%%%%%%%%%%%%%%%%%%%%%%%%%%
	%%%   Trabalhos Relacionados   %%%%
	%%%%%%%%%%%%%%%%%%%%%%%%%%%%%%%%%%%
	\section{TRABALHOS RELACIONADOS}
		\subfile{sections/trabalhos-relacionados.tex}
	\clearpage

	%%%%%%%%%%%%%%%%%%%%%%%%%%%%%%%%%%%
	%%%         Metodologia        %%%%
	%%%%%%%%%%%%%%%%%%%%%%%%%%%%%%%%%%%
	\section{METODOLOGIA}
		\subfile{sections/metodologia.tex}
	\clearpage

	%%%%%%%%%%%%%%%%%%%%%%%%%%%%%%%%%%%
	%%%  Resultados Experimentais  %%%%
	%%%%%%%%%%%%%%%%%%%%%%%%%%%%%%%%%%%
	\section{RESULTADOS EXPERIMENTAIS}

		\lipsum[1]

		\subsection{Performance}

			\lipsum[1]

		\subsection{Compatibilidade}

			\lipsum[1]

		\subsection{Acessibilidade}

			\lipsum[1]

		\subsection{Usabilidade}

			\lipsum[1]


	%%%%%%%%%%%%%%%%%%%%%%%%%%%%%%%%%%%
	%%%         Conclusão          %%%%
	%%%%%%%%%%%%%%%%%%%%%%%%%%%%%%%%%%%
	\section{CONCLUSÃO}

	\lipsum[1]

		\subsection{Trabalhos Futuros}

			\par Provavelmente fazer uma versão web

			\lipsum[1]

	\newpage


	%%%%%%%%%%%%%%%%%%%%%%%%%%%%%%%%%%%
	%%%        Referências         %%%%
	%%%%%%%%%%%%%%%%%%%%%%%%%%%%%%%%%%%
	\section*{REFERÊNCIAS}
	\addcontentsline{toc}{section}{REFERÊNCIAS}

	\printbibliography[heading=none]

	\newpage

	%%%%%%%%%%%%%%%%%%%%%%%%%%%%%%%%%%%
	%%%         Glossário          %%%%
	%%%%%%%%%%%%%%%%%%%%%%%%%%%%%%%%%%%
	\section*{GLOSSÁRIO}
	\addcontentsline{toc}{section}{GLOSSÁRIO}

	\begin{tabular}{p{3cm} p{0.6\textwidth}}
	  Software & \textit{Programa de Computador} [referenciar] \\
	  Dolor & \textit{Instituto Federal de Educação, Ciência e Tecnologia} \\
	  Sit Amet & \textit{Unified Modeling Language} \\
	\end{tabular}\\


	\newpage

	%%%%%%%%%%%%%%%%%%%%%%%%%%%%%%%%%%%
	%%%          Apêndice          %%%%
	%%%%%%%%%%%%%%%%%%%%%%%%%%%%%%%%%%%
	\section*{APÊNDICE}
	\addcontentsline{toc}{section}{APÊNDICE}

	\lipsum[1]

	\newpage

\end{document}
