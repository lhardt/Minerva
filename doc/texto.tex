\documentclass[12pt,a4paper]{article}
\usepackage[utf8]{inputenc}

\usepackage{ifpdf} % configuraç
\usepackage{lipsum} % lorem
\usepackage{scrextend} % addmargin
\usepackage{fontspec} % Arial
\usepackage[portuguese]{babel}
\usepackage{csquotes}
\usepackage[font={scriptsize}]{caption}
\usepackage{indentfirst} % indentação do primeiro parágrafo de seção
\usepackage{makeidx} % Títulos das tabelas
\usepackage[pagestyles]{titlesec} % titleformat
\usepackage{geometry}
\usepackage{amsfonts} % blackletter
\usepackage{graphicx} % resizebox
\usepackage{svg} % Grafos
\usepackage{enumitem} % HC01, HC02...
% citestyle authoryear para citações anbt
\usepackage[backend=biber,sorting=nty, style=abnt, citestyle=abnt-numeric]{biblatex}

\addto\captionsportuguese{
	\renewcommand{\contentsname}{ }
	\renewcommand{\contentsname}{ }
	\renewcommand{\listfigurename}{ }
	\renewcommand{\listtablename}{ }
}

\usepackage[breaklinks=true]{hyperref}


\title{MINERVA: UMA SOLUÇÃO INFORMATIZADA PARA O ESCALONAMENTO DE PROFESSORES NAS ESCOLAS BRASILEIRAS}
\ifpdf
	\hypersetup{
		pdftitle    = Minerva,
		pdfauthor   = {Léo Hardt, leom.hardt@gmail.com},
		pdfsubject  = TCC - Léo Hardt,
		pdfcreator  = Léo Hardt,
		pdfproducer = PDFLatex,
		pdfkeywords = {Software, Escola, Grade Horária}
	}
\fi

\geometry{
	a4paper,
	left=30mm,
	right=20mm,
	top=30mm,
	bottom=20mm
}

\pagestyle{myheadings}

\addbibresource{texto.bib}

\setmainfont{Arial}
\setlength{\parindent}{12.5mm}
\renewcommand{\baselinestretch}{1.5}

\titleformat{\section}
  {\normalfont\scshape\bfseries}{\thesection}{1em}{}
\titleformat{\subsection}
	{\normalfont\scshape\bfseries}{\thesubsection}{1em}{}
\titleformat{\subsubsection}
	{\normalfont\scshape}{\thesubsubsection}{1em}{}

\newenvironment{bottompar}{\par\vspace*{\fill}}{\clearpage}

\begin{document}



	%%%%%%%%%%%%%%%%%%%%%%%%%%%%%%%%%%%
	%%%            Capa            %%%%
	%%%%%%%%%%%%%%%%%%%%%%%%%%%%%%%%%%%
	\thispagestyle{empty}

	\begin{center}
		INSTITUTO FEDERAL DE EDUCAÇÃO, CIÊNCIA E TECNOLOGIA DO RIO GRANDE DO SUL - CAMPUS CANOAS \\
		CURSO TÉCNICO DE DESENVOLVIMENTO DE SISTEMAS INTEGRADO AO ENSINO MÉDIO\\
	\end{center}

	\vskip 3cm

	\begin{center}
		LÉO MARCO DE ASSIS HARDT
	\end{center}

	\vskip 5cm

	\begin{center}
		\textbf{MINERVA: UMA SOLUÇÃO INFORMATIZADA PARA O ESCALONAMENTO DE PROFESSORES NAS ESCOLAS BRASILEIRAS}
	\end{center}


	\begin{bottompar}
		\begin{center}
		CANOAS \\
		2020
		\end{center}
	\end{bottompar}

	%%%%%%%%%%%%%%%%%%%%%%%%%%%%%%%%%%%
	%%%       Folha de Rosto       %%%%
	%%%%%%%%%%%%%%%%%%%%%%%%%%%%%%%%%%%
	\thispagestyle{empty}

	\begin{center}
		LÉO MARCO DE ASSIS HARDT
	\end{center}

	\vskip 3cm


	\begin{center}
		\textbf{MINERVA: UMA SOLUÇÃO INFORMATIZADA PARA O ESCALONAMENTO DE PROFESSORES NAS ESCOLAS BRASILEIRAS}
	\end{center}

	\vskip 2cm

	\begin{addmargin}[7.5cm]{0em}

		\setlength{\parindent}{0mm}
		Trabalho de Conclusão de Curso apresentado como requisito parcial para obtenção do grau de Técnico em Desenvolvimento de Sistemas pelo Instituto Federal de Educação, Ciência e Tecnologia do Rio Grande do Sul – Campus Canoas.\\

		Prof. Gustavo Neuberger\\
		Orientador

	\end{addmargin}

	\begin{bottompar}
		\begin{center}
			CANOAS \\
			2020
		\end{center}
	\end{bottompar}


	%%%%%%%%%%%%%%%%%%%%%%%%%%%%%%%%%%%
	%%%       Agradecimentos       %%%%
	%%%%%%%%%%%%%%%%%%%%%%%%%%%%%%%%%%%
	\thispagestyle{empty}
	\section*{AGRADECIMENTOS}

	Gostaria de agradecer, em primeiro lugar, a todos que me apoiaram diretamente. Colegas, amigos, parentes e professores. Em segundo lugar, a todos aqueles que acreditaram num ensino de qualidade e disponibilizaram seu conhecimento ao mundo, mesmo que nunca se conheça o total impacto de tais ações. Nomes dos quais me lembro são Richard Feynmann, Grant Sanderson e Sal Khan. Por fim, gostaria de agradecer aos responsáveis pelo ambiente no qual tive a honra de me formar e no qual gostaria que muitos outros alunos tivessem a mesma oportunidade de evoução que eu tive.

	\newpage


	%%%%%%%%%%%%%%%%%%%%%%%%%%%%%%%%%%%
	%%%     Resumo em Português    %%%%
	%%%%%%%%%%%%%%%%%%%%%%%%%%%%%%%%%%%
	\thispagestyle{empty}
	\section*{RESUMO}

	No presente trabalho, é estudado o problema da elaboração do cronograma semanal escolar. São catalogadas necessidades e preferências de escolas em relação ao horário final, bem como carências de métodos atualmente utilizados. São consideradas soluções anteriores, sejam elas computacionais ou não. É identificada, então, uma necessidade das escolas brasileiras por um \textit{software} público e de fácil utilização para a elaboração do mesmo. Um software com esse fim é implementado utilizando uma abordagem de criação de horários semelhante à usada em motores de xadrez. Essa abordagem é comparada	computacional e utilitariamente com abordagens anteriores. O software final é testado e é feita uma pesquisa de satisfação com as escolas que participaram do teste. Os resultados são mostrados. São listados possíveis trabalhos futuros na área.


	\begingroup
		\setlength{\parindent}{0mm}
		\textbf{Palavras chave:} \textit{Software}; Escola; Grade Horária.
	\endgroup
	\newpage


	%%%%%%%%%%%%%%%%%%%%%%%%%%%%%%%%%%%
	%%%      Resumo em Inglês      %%%%
	%%%%%%%%%%%%%%%%%%%%%%%%%%%%%%%%%%%
	\thispagestyle{empty}
	\section*{ABSTRACT}

	In the present work, the school timetabling problem is studied. School necessities and preferences about the final timetable, as well as deficiencies of the current methods are listed. Previous solutions, be they computational or not, are considered. A necessity by the brazilian schools for an easy to use public software to create timetables is identified. A software is implemented with this goal, using an approach similar to that used in chess engines. This approach is compared from computational and utilitarian perspectives. The software is tested, and a satisfaction survey is made for schools which participated in the test. The results are shown. Possible future work in the area is listed.

	\begingroup
		\setlength{\parindent}{0mm}
		\textbf{Keywords:} Software; School; Timetable.
	\endgroup

	\newpage


	%%%%%%%%%%%%%%%%%%%%%%%%%%%%%%%%%%%
	%%%      Resumo em Alemão      %%%%
	%%%%%%%%%%%%%%%%%%%%%%%%%%%%%%%%%%%
	\thispagestyle{empty}
	\section*{ZUSAMMENFASSUNG}

	In der vorliegenden Arbeit, ist das Stundenplanungsproblem untergesucht. Schulnotwendigkeiten und Vorlieben, sowie Mängel der aktuellen Systeme sind aufgelistet. Verherige Lösungen, rechnerisch oder nicht, sind betracht.

	\lipsum[1]


	\begingroup
		\setlength{\parindent}{0mm}
		\textbf{Schlagworter:} Wort1; Wort2; Wort3.
	\endgroup

	\newpage

	%%%%%%%%%%%%%%%%%%%%%%%%%%%%%%%%%%%
	%%%      Lista de Tabelas      %%%%
	%%%%%%%%%%%%%%%%%%%%%%%%%%%%%%%%%%%
	\thispagestyle{empty}
	\section*{LISTA DE TABELAS}

	\listoftables

	\newpage

	%%%%%%%%%%%%%%%%%%%%%%%%%%%%%%%%%%%
	%%%      Lista de Figuras      %%%%
	%%%%%%%%%%%%%%%%%%%%%%%%%%%%%%%%%%%
	\thispagestyle{empty}
	\section*{LISTA DE FIGURAS}

	\listoffigures

	\newpage


	%%%%%%%%%%%%%%%%%%%%%%%%%%%%%%%%%%%
	%%%      Lista de Siglas       %%%%
	%%%%%%%%%%%%%%%%%%%%%%%%%%%%%%%%%%%
	\thispagestyle{empty}
	\section*{LISTA DE ABREVIATURAS E SIGLAS}

	\begin{tabular}{p{3cm} p{0.6\textwidth}}
	  C & \textit{C Programming Language} \\
	  IFRS & \textit{Instituto Federal de Educação, Ciência e Tecnologia do Rio Grande do Sul} \\
	  UML & \textit{Unified Modeling Language} \\
	  HC & \textit{Hard Constraint} \\
	  SC & \textit{Soft Constraint} \\
	  OC & \textit{Optional Constraint}
	\end{tabular}\\


	\newpage


	%%%%%%%%%%%%%%%%%%%%%%%%%%%%%%%%%%%
	%%%     Lista de Algorítmos    %%%%
	%%%%%%%%%%%%%%%%%%%%%%%%%%%%%%%%%%%
	\thispagestyle{empty}
	\section*{LISTA DE ALGORÍTMOS}

	% \listofalgorithms

	\newpage



	%%%%%%%%%%%%%%%%%%%%%%%%%%%%%%%%%%%
	%%%          Sumário           %%%%
	%%%%%%%%%%%%%%%%%%%%%%%%%%%%%%%%%%%
	\thispagestyle{empty}
	\section*{SUMÁRIO}

	\begingroup
		\let\clearpage\relax
		\vspace{-1cm} % Removed title space.
		\tableofcontents
	\endgroup

	\newpage


	%%%%%%%%%%%%%%%%%%%%%%%%%%%%%%%%%%%
	%%%         Introdução         %%%%
	%%%%%%%%%%%%%%%%%%%%%%%%%%%%%%%%%%%
	\section{INTRODUÇÃO}



		\par Elaborar um cronograma é uma tarefa extraordinária. Mesmo nos casos mais simples, como nos cronogramas pessoais, há um grande número de possiblidades, restrições e preferências do usuário. Assim, a dificuldade de produzir um cronograma eficaz aumenta rapidamente conforme seu tamanho.

		% Assim, a qualidade do cronograma depende não só das restrições seguidas, mas também do grau de satisfação dessas preferências.
		\par Mesmo assim, eles são de vital importância para a rotina de indivíduos, escolas, indústrias, aeroportos, hospitais e eventos esportivos. Instituições de ensino, em particular, têm custos enormes com a elaboração de suas grades horárias: elaborá-las manualmente pode demandar dias, ou até semanas \cite{appleby,nikita} de um profissional. No entanto, esse processo é rotineiro, já que a cada mudança na docência, um novo cronograma pode ser necessário.

		\par Uma grade horária mal-pensada pode prejudicar e muito uma escola. Pode-se imaginar, por exemplo, que esta faça um professor frequentar a escola desnecessariamente, ou que uma aula geminada é dividida ao meio pelo intervalo e seu rendimento, reduzido. Ou que, por descuido de seu elaborador, a grade horária requisita um professor em duas turmas ao mesmo tempo.

		\par Processos de verificação extensiva para esses casos são facilmente automatizados por computador. Não só isso, mas podem ser percorridas formas de otimização da grade horária em velocidades incomparáveis às de qualquer humano, livrando a escola de custos e desafogando um dos processos mais lentos da admistração acadêmica.

		\par Mas para a administração escolar, não é fácil encontrar um software que acomode suas necessidades. Aulas tripas, com dois professores, logo antes do intervalo são exemplos exigências muito específicas, às quais muitos programas não foram pensados para atender \cite{nikita, carter1995}.

		\par Por ser mais escasso e mais difícil de criar, o preço de um software comercial que atenda às especificadades de uma escola é elevado. Escolas de pequeno porte, portanto, muitas vezes não têm verba suficiente para utilizá-los, optando pela criação manual e aceitando a ineficácia que isso pode causar. Mesmo nas escolas com tal capital, se possível, seria ideal invesí-lo na manutenção de infraestrutura, da docência, da merenda, e melhorar assim, a qualidade de ensino.

		%%%%%%%%%%%%%%%%%%%%%%%%%%%%%%%%%%%
		%%%      Solução Proposta      %%%%
		%%%%%%%%%%%%%%%%%%%%%%%%%%%%%%%%%%%
		\subsection{Solução Proposta}

			 \par A partir das considerações acima, constata-se a ausência de um software público \cite{publico} que supra as necessidades das escolas brasileiras e que seja de fácil utilização por professores de fora da área de informática. Tal software, então, deveria poderia ser implementado e utilizado em computadores da rede pública de ensino -- ou seja, compatível com os sistemas operacionais Windows, Ubuntu e Linux Educacional \cite{proinfo, w3s}.

			 \par Em tal software, o professor responsável pela criação do horário escolar insere as necessidades programáticas de aula, as demandas subjetivas dos professores e os horários disponíveis para esses encontros. O sistema então, em constante interação com o usuário, cria o horário escolar.  Desta forma, são reduzidos custos em tempo e em dinheiro da escola que se propõe a utilizar o sistema.

		%%%%%%%%%%%%%%%%%%%%%%%%%%%%%%%%%%%
		%%%      Objetivos Gerais      %%%%
		%%%%%%%%%%%%%%%%%%%%%%%%%%%%%%%%%%%
		\subsection{Objetivos Gerais}

			\par Implementar um software público para a facilitação da criação e manutenção de cronograma escolar.


		%%%%%%%%%%%%%%%%%%%%%%%%%%%%%%%%%%%
		%%%   Objetivos Específicos    %%%%
		%%%%%%%%%%%%%%%%%%%%%%%%%%%%%%%%%%%
		\subsection{Objetivos Específicos}

			\begin{itemize}
				\item Analisar a literatura existente em relação a softwares de escalonamento e \textit{timetabling};
				\item Analistar a literatura existente em relação à criação de boas interfaces do usuário;
				\item Realizar um levantamento de softwares da área, estabelecendo métricas de comparação tendo em vista a experiência do usuário final;
				\item Definir os requisitos de uma grade horária de forma abrangente e precisa;
				\item Definir os requisitos de um sistema que gere tais grades horárias;
				\item Realizar a modelagem do sistema;
				\item Implementar um sistema leve, eficiente e de fácil utilização para a criação de grades horárias escolares;
				\item Testar o sistema, realizando \textit{benchmarking} e graduação do horário gerado, com\-pa\-ran\-do-o com soluções anteriores;
				\item Publicar o software produzido no repositório de Software Público do governo brasileiro;
				\item Documentar o processo.
			\end{itemize}

	\newpage


	%%%%%%%%%%%%%%%%%%%%%%%%%%%%%%%%%%%
	%%%   Fundamentação Teórica    %%%%
	%%%%%%%%%%%%%%%%%%%%%%%%%%%%%%%%%%%
	\section{FUNDAMENTAÇÃO TEÓRICA}

		\par A criação de grades horárias é um campo em desenvolvimento \cite{patat2020}. Há mais de 50 anos são pesquisadas formas de criação de cronogramas escolares com auxílio computacional \cite{appleby}. Há, então, uma abundância de trabalhos relacionados. Citamos aqui três abordagens gerais tomadas para a criação do cronograma:

		\begin{itemize}
			\item \textit{Constraint Logic Programming} (CLP): o problema é conceituado como uma lista de variáveis, com domínios e restrições. \cite{badoni} De modo geral, pode-se dizer que algorítmos tanto de busca de soluções quanto de propagação de restrições são utilizados \cite{citar_alguém}.
			\item Solução por \textit{Clusters}: o grupo total de aulas é dividido em \textit{clusters}, conflitantes entre si (não podem acontecer no mesmo período), mas que não são conflitantes internamente. O programa tenta, então, assinalar um período para cada cluster. Este tipo de solução é criticado pois a fixação de grupos acontece muito cedo e dificulta a otimização dentro dos grupos. \cite{muller}
			\item Soluções Meta-Heurísticas: neste tipo de solução, inicia-se com um horário gerado aleatoriamente e busca-se a otimização de uma função objetivo, por meio de estratégias de busca. Estão incluídos aqui Tabu Search, Simulated Annealing e as soluções genéticas.
		\end{itemize}

		\subsection{Descrição Formal do Cronograma}

			\par Podemos definir o cronograma da seguinte forma:

			\begin{itemize}
				\item $P = \{p_1, p_2, p_3, ..., p_\pi\}$ é a lista de todos os $\pi$ períodos;
				\item $T = \{t_1, t_2, t_3, ..., t_\theta\}$ é a lista de todos os $\theta$ professores;
				\item $C = \{c_1, c_2, c_3, ..., c_\kappa\} $ é a lista de todas as $\kappa$ turmas;
				\item $R = \{r_1, r_2, r_3, ..., r_\rho\}$ é a lista de todas as $\rho$ salas;
				\item $L = \{l_1, l_2, l_3, ..., l_\lambda\}$ é a lista de todas as $\lambda$ aulas, sendo uma aula uma tupla $l = (p,t,c,r)$;
			\end{itemize}

			\par Os requisitos aqui mencionados são divididos em três categorias: rígidas, quando o requisito deve ser seguido por qualquer cronograma; flexíveis, que não invalidam a solução final se quebradas; e opcional, quando o usuário opta se o cronograma deve segui-lo ou não. Os requisitos opcionais podem ser tanto rígidas quanto flexíveis. Cada um desses requisitos, na implementação, será utilizado para eliminar possiblidades no cronograma.

			\textbf{Pressuposições (ASs):}
			\begin{enumerate}
				\item O cronograma abrange um Ciclo, que é dividido em Dias e Períodos;
				\item Há um número fixo de dias em um ciclo;
				\item Há um número fixo de períodos em um dia, que é o mesmo para qualquer outro dia;
				\item Todos os períodos tem igual tamanho;
				\item O currículo de uma turma para cada disciplina é um número exato de períodos por ciclo;
			\end{enumerate}

			\textbf{Requisitos Rígidos (HCs):}
			\begin{enumerate}
				\item Um professor pode ministrar, no máximo, uma aula por período;
				\item Uma turma pode assistir, no máximo, a duas aulas por período;
				\item Uma sala pode ser ocupada por, no máximo, uma aula por período;
				\item O número de períodos utilizados para uma disciplina por uma turma em um ciclo deve ser o especificado no currículo escolar;
				\item Um único grupo de professores é responsável por ministrar todas as aulas de uma disciplina de uma turma, e deve se fazer presente em todas essas aulas;
				\item Nenhuma sala será ocupada acima de sua capacidade;
				\item Nenhuma aula será dada em uma sala que não tenha as características necessárias;
				\item Uma aula só será ministrada por um professor (ou grupo de professores) que está capacitado para tal.
				\item Um professor não ministrará aulas em uma sala em que ele não pode entrar;
				\item Uma turma não ministrará aulas em uma sala em que ela não pode entrar;
			\end{enumerate}

			\textbf{Requisitos Flexíveies (OCs):}
			\begin{enumerate}
				\item As preferências de aula de cada turma, considerando o período em que a aula é dada;
				\item As preferências de aula de cada professor, considerando o período em que a aula é dada;
				\item As preferências de aula de cada par professor-disciplina, considerando o período em que a aula é dada;
				\item As preferências de geminação de alguns pares professor-disciplina;
				\item Podem ser definidas ordens de preferência por cada professor para o uso de salas para cada característica;
			\end{enumerate}
;
			\textbf{Requisitos Opcionais Rígidos (HOCs):}
			\begin{enumerate}
				\item A escola pode escolher operar em um subconjunto dos períodos de um ciclo;
				\item O professor poderá ter um conjunto de períodos para a elaboração de suas aulas;
				\item Algumas aulas poderão ser previamente fixadas em alguns horários;
				\item Algumas professores poderão ser previamente fixados para ministrar aulas de algumas turmas;
				\item Algumas salas poderão ser previamente fixadas para abrigar algumas aulas;
				\item Algumas turmas podem ter um número máximo de aulas por dia por grupos de disciplina;
				\item Alguns professores podem ter um número máximo de aulas por dia por turmas (mas sem discriminar qual turma);
				\item Pode haver, para cada turma, um horário fixo de entrada e de saída da instituição;
				\item Pode haver, para cada turma, a possibilidade de períodos vagos entre aulas;
				\item Uma sala pode ter disponibilidade limitada durante um ciclo;


			\end{enumerate}



	\newpage

	%%%%%%%%%%%%%%%%%%%%%%%%%%%%%%%%%%%
	%%%   Trabalhos relacionados   %%%%
	%%%%%%%%%%%%%%%%%%%%%%%%%%%%%%%%%%%
	\section{TRABALHOS RELACIONADOS}

		Esta seção tem como finalidade comparar o trabalho

		\par Para simplificar a comparação com softwares similares, fica estabelecida a métrica de $50$ turmas com as quais o sistema deve lidar. Esse número fica muito acima da média de turmas por escola de cidades como Porto Alegre, Canoas e Cachoeirinha, por exemplo \cite{inep}. Fica viável, então, comparar preços de softwares comerciais, dado que estes geralmente cobram baseado no número de turmas.

		\par Desta miríade, pode-se destacar aSc Timetables \cite{rel_asctimetables}, Horário Fácil \cite{rel_horariofacil}, Urânia \cite{rel_urania} e UniTime \cite{rel_unitime}. Neste capítulo, estes softwares serão apresentados, afim de traçar melhor as necessidades que os softwares da área geralmente têm. https://help.asctimetables.com/text.php?id=592\&lang=pt

		%%%%%%%%%%%%%%%%%%%%%%%%%%%%%%%%%%%
		%%%       aSc Timetables       %%%%
		%%%%%%%%%%%%%%%%%%%%%%%%%%%%%%%%%%%
		\subsection{aSc Timetables}

			\par Segundo sua própria página, aSc Timetables é um software utilizado por 150.000 escolas. Tanto o programa quanto a \textit{webpage} podem ser utilizadas em 13 línguas diferentes. Isso dá a entender que o software se alinha com as demandas apresentadas por uma gama muito grande de escolas. Para escolas brasileiras, seu preço base é de USD 120.

			\par Algumas características que podemos destacar são: ajustes manuais; detecção automática de conflitos no horário; entrada simples de dados;  o horário pode ser acessado por dispositivos móveis; importação de dados;

		% \centering
		\begin{table}[htb]
				\begin{center}
				\scalebox{0.7}{
					\begin{tabular}{| l | c | c | c | c | c | c | c | c | }
						\hline
						Nome & Português & Gratuito & Livre & Ativo & Local & Online & Ref   \\
						\hline\hline
						%%%%%%%%%%%%%%%%%%  POR GRA LIV ATI LOC ONL REF %%%%%%%%%%%%%%%%%%%%%%%%%%%%%%%%%%%%%%%%%%
						UniTime  			& N & S & S & S & S & N & \cite{rel_unitime}  						\\ \hline
						Urânia 		   	 	& S & N & N & S & S & N & \cite{rel_urania} 						\\ \hline
						Horário Fácil 	  	& S & N & N & S & N & S & \cite{rel_horariofacil} 					\\ \hline
						GridClass 	    	& S & N & N & S & N & S & \cite{rel_gridclass} 						\\ \hline
						Benchmark Timetable & N & N & N & ? & S & N & \cite{rel_benchmark,rel_supertimetable} 	\\ \hline
						ASC Timetables 		& S & N & N & S & S & ? & \cite{rel_asctimetables}					\\ \hline
						Untis 	 			& N & N & N & S & S & ? & \cite{rel_untis} 							\\ \hline
						Lantiv 				& N & N & N & S & * & * & \cite{rel_lantiv} 						\\ \hline
						Timetabler          & N & N & N & S & S & N & \cite{rel_timetabler} 					\\ \hline
						Nova T6             & ? & ? & N & ? & N & ? & \cite{rel_novat6}							\\ \hline
						iTimetable          & N & N & N & N & N & **& \cite{rel_itimetable}  					\\ \hline
						Timetable Web 	    & S & N & N & ? & N & S & \cite{rel_timetableweb} 					\\ \hline
						Make Your Timetable & ? & S & N & ? & N & S & \cite{rel_makeyourtimetable}	 			\\ \hline
						School Softwares    & N & N & N & S & S & N & \cite{rel_schoolsoftwares} 				\\ \hline
						TimeFinder          & N & S & S & N & S & N & \cite{rel_timefinder}						\\ \hline
						FET 				& N & S & S & S & S & N & \cite{rel_fet}							\\ \hline
						Peñalara GHC		& S & S & N & S & S & N & \cite{rel_penalara}						\\ \hline
						CMIS	  			& N & N & N & ? & N & S & \cite{rel_penalara}						\\ \hline
						%%%%%%%%%%%%%%%%%%%%%%%%%%%%%%%%%%%%%%%%%%%%%%%%%%%%%%%%%%%%%%%%%%%%%%%%%%%%%%%%%%%%%%%%%%
					\end{tabular}
				}
				\caption{Comparativo entre softwares comerciais da área}
			\end{center}
		\end{table}
		* - O programa é local, mas os dados ficam na nuvem.
		** - A grade horária é encomendada.
		\\

		Há muito desenvolvimento na área. Em particular, são notórias as fontes:

		\begin{itemize}
			\item Conferência PATAT -- Practice and Theory of Automated Timetabling ;
			\item ITC -- International Timetabling Competition;
			\item SBPO -- Simpósio Brasileiro de Pesquisa Operacional;
		\end{itemize}

	\newpage

	%%%%%%%%%%%%%%%%%%%%%%%%%%%%%%%%%%%
	%%%         Metodologia        %%%%
	%%%%%%%%%%%%%%%%%%%%%%%%%%%%%%%%%%%
	\section{METODOLOGIA}

		%%%%%%%%%%%%%%%%%%%%%%%%%%%%%%%%%%%
		%%%    Tecnologias Adotadas    %%%%
		%%%%%%%%%%%%%%%%%%%%%%%%%%%%%%%%%%%
		\subsection{Tecnologias Adotadas}

			\par Pelo relativo baixo nível e pela eficiência que ainda retém boas abstrações, foram escolhidas as linguagens C e C++ para o desenvolvimento do projeto. Foi escrito em C o código que faz o cronograma. A multiplicidade de línguas se dá por que os compiladores podem realizar otimizações mais agressivas na linguagem C, mas é em C++ que geralmente encontra-se bibliotecas e frameworks de interface do usuário.

			\par Pelo fato de ser multiplataforma -- satisfazendo a compatibilidade entre Windows, Linux Educacional e Ubuntu -- foi escolhida a biblioteca wxWidgets para o desenvolvimento da GUI.

			\par Pela leveza que ainda retém eficiência e variedade, foi escolhida a SQLite3 como biblioteca para criar manusear o banco dados.

		%%%%%%%%%%%%%%%%%%%%%%%%%%%%%%%%%%%
		%%%    Ferramentas Adotadas    %%%%
		%%%%%%%%%%%%%%%%%%%%%%%%%%%%%%%%%%%
		\subsection{Ferramentas Adotadas}

			\par As principais ferramentas adotas foram: brModelo e astah na modelagem; gcc na compilação em sistemas GNU/Linux; mingw na compilação em Windows; valgrind no debbuging; Atom na edição de texto; e \LaTeX \, na escrita do texto.

		%%%%%%%%%%%%%%%%%%%%%%%%%%%%%%%%%%%
		%%%    Modelagem do Sistema    %%%%
		%%%%%%%%%%%%%%%%%%%%%%%%%%%%%%%%%%%
		\subsection{Modelagem do Sistema}

			A solução proposta pode ser dita como \textit{exaustiva}, no sentido de que dados tempo e memória suficientes, pode-se encontrar a melhor solução

			É utilizada uma Decision Tree para a seleção de períodos, horários e professores no sistema.

			\lipsum[1]


			%%%%%%%%%%%%%%%%%%%%%%%%%%%%%%%%%%%
			%%%        Casos de Uso        %%%%
			%%%%%%%%%%%%%%%%%%%%%%%%%%%%%%%%%%%
			\subsubsection{Casos de Uso}

				\lipsum[1]

			%%%%%%%%%%%%%%%%%%%%%%%%%%%%%%%%%%%
			%%%       Banco de Dados       %%%%
			%%%%%%%%%%%%%%%%%%%%%%%%%%%%%%%%%%%
			\subsubsection{Banco de Dados}

				\lipsum[1]

			%%%%%%%%%%%%%%%%%%%%%%%%%%%%%%%%%%%
			%%%          Algoritmos        %%%%
			%%%%%%%%%%%%%%%%%%%%%%%%%%%%%%%%%%%
			\subsubsection{Algoritmos}

				\textbf{Orientações Gerais}

				\par Sequential method of Saturation Degree (um dos mais robustos segundo Carter, Laporte e Lee 1995. Ler o artigo!).
				Lewis 2015 também vê resultados mais favoráveis no Saturation

				\par Ler sobre 1-opt e 2-opt (Carter, Laporte e Chinnek)

				\par Usar alguma aleatoridade depois do primeiro cronograma para dar mais opções para o usuário



	\section{RESULTADOS EXPERIMENTAIS}

		\lipsum[1]

		\subsection{Performance}

			\lipsum[1]

		\subsection{Compatibilidade}

			\lipsum[1]

		\subsection{Acessibilidade}

			\lipsum[1]

		\subsection{Usabilidade}

			\lipsum[1]


	%%%%%%%%%%%%%%%%%%%%%%%%%%%%%%%%%%%
	%%%         Conclusão          %%%%
	%%%%%%%%%%%%%%%%%%%%%%%%%%%%%%%%%%%
	\section{CONCLUSÃO}

	\lipsum[1]

		\subsection{Trabalhos Futuros}

			\lipsum[1]

	\newpage


	%%%%%%%%%%%%%%%%%%%%%%%%%%%%%%%%%%%
	%%%        Referências         %%%%
	%%%%%%%%%%%%%%%%%%%%%%%%%%%%%%%%%%%
	\section*{REFERÊNCIAS}
	\addcontentsline{toc}{section}{REFERÊNCIAS}

	\printbibliography[heading=none]

	\newpage

	%%%%%%%%%%%%%%%%%%%%%%%%%%%%%%%%%%%
	%%%         Glossário          %%%%
	%%%%%%%%%%%%%%%%%%%%%%%%%%%%%%%%%%%
	\section*{GLOSSÁRIO}
	\addcontentsline{toc}{section}{GLOSSÁRIO}

	\begin{tabular}{p{3cm} p{0.6\textwidth}}
	  Software & \textit{Programa de Computador} [referenciar] \\
	  Dolor & \textit{Instituto Federal de Educação, Ciência e Tecnologia} \\
	  Sit Amet & \textit{Unified Modeling Language} \\
	\end{tabular}\\


	\newpage

	%%%%%%%%%%%%%%%%%%%%%%%%%%%%%%%%%%%
	%%%          Apêndice          %%%%
	%%%%%%%%%%%%%%%%%%%%%%%%%%%%%%%%%%%
	\section*{APÊNDICE}
	\addcontentsline{toc}{section}{APÊNDICE}

	\lipsum[1]

	\newpage

\end{document}
