\documentclass[12pt,a4paper]{article}
\usepackage[utf8]{inputenc}

\usepackage{ifpdf} % configuraç
\usepackage{lipsum} % lipsum
\usepackage{scrextend} % addmargin
\usepackage{fontspec} % Arial
\usepackage[portuguese]{babel}
 \usepackage{indentfirst} % indentação do primeiro parágrafo de seção
\usepackage{makeidx} % Títulos das tabelas
\usepackage[pagestyles]{titlesec} % titleformat
\usepackage{geometry}
\usepackage{graphicx} % resizebox
% citestyle authoryear para citações anbt
\usepackage[backend=biber,sorting=nty, style=abnt, citestyle=abnt-numeric]{biblatex}

\addto\captionsportuguese{
	\renewcommand{\contentsname}{ }
	\renewcommand{\contentsname}{ }
	\renewcommand{\listfigurename}{ }
	\renewcommand{\listtablename}{ }
}

\usepackage[breaklinks=true]{hyperref}
\addbibresource{texto.bib}


\title{MINERVA: UMA SOLUÇÃO INFORMATIZADA PARA O ESCALONAMENTO DE PROFESSORES NAS ESCOLAS BRASILEIRAS}
\ifpdf
	\hypersetup{
		pdftitle    = Minerva,
		pdfauthor   = {Léo Hardt, leom.hardt@gmail.com},
		pdfsubject  = TCC - Léo Hardt,
		pdfcreator  = Léo Hardt,
		pdfproducer = PDFLatex,
		pdfkeywords = {Software, Escola, Grade Horária}
	}
\fi

\geometry{
	a4paper,
	left=30mm,
	right=20mm,
	top=30mm,
	bottom=20mm
}

\pagestyle{myheadings}

\setmainfont{Arial}
\setlength{\parindent}{12.5mm}
\renewcommand{\baselinestretch}{1.5}

\titleformat{\section}
  {\normalfont\scshape\bfseries}{\thesection}{1em}{}
\titleformat{\subsection}
	{\normalfont\scshape}{\thesubsection}{1em}{}

\newenvironment{bottompar}{\par\vspace*{\fill}}{\clearpage}

\begin{document}



	%%%%%%%%%%%%%%%%%%%%%%%%%%%%%%%%%%%
	%%%            Capa            %%%%
	%%%%%%%%%%%%%%%%%%%%%%%%%%%%%%%%%%%
	\thispagestyle{empty}

	\begin{center}
		INSTITUTO FEDERAL DE EDUCAÇÃO, CIÊNCIA E TECNOLOGIA DO RIO GRANDE DO SUL - CAMPUS CANOAS \\
		CURSO TÉCNICO DE DESENVOLVIMENTO DE SISTEMAS INTEGRADO AO ENSINO MÉDIO\\
	\end{center}

	\vskip 3cm

	\begin{center}
		LÉO MARCO DE ASSIS HARDT
	\end{center}

	\vskip 5cm

	\begin{center}
		\textbf{APÊNDICE II - MINERVA: UMA SOLUÇÃO INFORMATIZADA PARA O ESCALONAMENTO DE PROFESSORES NAS ESCOLAS BRASILEIRAS}
	\end{center}

	\vskip 5cm

	\begin{center}
		Orientador: Gustavo Neuberger
	\end{center}



	\begin{bottompar}
		\begin{center}
		CANOAS \\
		2020
		\end{center}
	\end{bottompar}

	%%%%%%%%%%%%%%%%%%%%%%%%%%%%%%%%%%%
	%%%        Justificativa       %%%%
	%%%%%%%%%%%%%%%%%%%%%%%%%%%%%%%%%%%
    \section{JUSTIFICATIVA DO PROJETO CIENTÍFICO}

		\subsection{DESCRIÇÃO DO PROBLEMA}


			\par Elaborar um cronograma é uma tarefa extraordinária. Mesmo nos casos mais simples, como nos cronogramas pessoais, há um grande número de possiblidades, restrições e preferências do usuário. Assim, a dificuldade de produzir um cronograma eficaz aumenta rapidamente conforme seu tamanho.

			\par Mesmo assim, eles são de vital importância para a rotina de indivíduos, escolas, indústrias, aeroportos, hospitais e eventos esportivos. Instituições de ensino, em particular, tem custos enormes com a elaboração de suas grades horárias: estima-se que elaborá-las manualmente pode demandar centenas de horas \cite{appleby} de um profissional. No entanto, esse processo é rotineiro: a cada mudança na docência, um novo cronograma pode ser necessário

			\par Uma grade horária mal-pensada pode prejudicar e muito uma escola. Pode-se imaginar, por exemplo, que esta faça um professor frequentar a escola em um dia a mais, ou que uma aula de física foi cortada ao meio pelo intervalo e seu rendimento foi reduzido. Ou que, por descuido de seu elaborador, a grade horária requisita um professor em duas turmas ao mesmo tempo.

			\par Processos de verificação extensiva para esses casos são facilmente automatizados por computador. Não só isso, mas podem ser percorridas formas de otimização da grade horária em velocidades incomparáveis às de qualquer humano, livrando a escola de custos e desafogando um dos processos mais lentos da admistração acadêmica.

			% \par Até mesmo em instituições que o criam manualmente, um sistema computadorizado ainda pode filtrar possibilidades e auxiliar na visualização. Em outras, uma geração automatizada pode ser preferível. Em todo caso, pode-se tirar proveito de computadores.

			\par Mas para a administração escolar, não é fácil encontrar um software que acomode suas necessidades. Aulas geminadas, com dois professores, logo antes do intervalo são exemplos exigências muito específicas, às quais muitos programas não foram pensados para atender \cite{nikita}.

			\par O preço de um software que atende a essas considerações é maior, pois sua oferta é menor. Escolas de pequeno porte, portanto, muitas vezes não têm verba suficiente para utilizá-los, optando novamente pela criação manual. Mesmo nas escolas que têm capital, o ideal seria invesí-lo na manutenção de infraestrutura, da docência, da merenda, melhorando assim, a qualidade de ensino.

		\subsection{PROPOSTA DE SOLUÇÃO}

			 \par A partir das considerações acima, constata-se a ausência de um software público \cite{publico} que supra as necessidades das escolas brasileiras e que seja de fácil utilização por professores de fora da área de informática. Tal software, então, deveria poderia ser implementado e utilizado em computadores da rede pública de ensino -- ou seja, compatível com os sistemas operacionais Windows, Ubuntu e Linux Educacional \cite{proinfo}.

			 \par Em tal software, o professor responsável pela criação do horário escolar insere as necessidades programáticas de aula, as demandas subjetivas dos professores e os horários disponíveis para esses encontros. O sistema então, em constante interação com o usuário, cria o horário escolar.  Desta forma, são reduzidos custos em tempo e em dinheiro da escola que se propõe a utilizar o sistema.

		\subsection{OBJETIVO}

			\subsubsection{OBJETIVO GERAL}

				\par Implementar um software público para a facilitação da criação e manutenção de cronograma escolar.

			\subsubsection{OBJETIVOS ESPECÍFICOS}

				\begin{itemize}
					\item Analisar a literatura existente em relação a softwares de escalonamento e \textit{timetabling};
					\item Analistar a literatura existente em relação à criação de boas interfaces do usuário;
					\item Realizar um levantamento de softwares da área, estabelecendo métricas de comparação tendo em vista a experiência do usuário final;
					\item Definir os requisitos de uma grade horária de forma abrangente e precisa;
					\item Definir os requisitos de um sistema que gere tais grades horárias;
					\item Realizar a modelagem do sistema;
					\item Implementar um sistema leve, eficiente e de fácil utilização para a criação de grades horárias escolares;
					\item Testar o sistema, realizando \textit{benchmarking} e graduação do horário gerado, comparando-o com soluções anteriores;
					\item Publicar o software produzido no repositório de Software Público do governo brasileiro;
					\item Documentar o processo.
				\end{itemize}

	\section{TRABALHOS RELACIONADOS}

		A criação de grades horárias é um campo em desenvolvimento \cite{patat2020}. Há mais de 50 anos \cite{appleby} são pesquisadas formas de criação de cronogramas escolares com auxílio computacional. São citados resultados importantes ao longo do texto. Também encontra-se uma abundância de softwares comerciais com este fim.



	\section{CRONOGRAMA}

		\lipsum[1]

		\\

		\scalebox{0.7}{
			\begin{tabular}{| c | c *{10}{ c | c } c|}
				\hline
				Nome da Atividade
				 & \multicolumn{2}{c}{Fev} & \multicolumn{2}{c}{Mar}
				 & \multicolumn{2}{c}{Abr} & \multicolumn{2}{c}{Mai} & \multicolumn{2}{c}{Jun}
				 & \multicolumn{2}{c}{Jul} & \multicolumn{2}{c}{Ago} & \multicolumn{2}{c}{Set}
				 & \multicolumn{2}{c}{Out} & \multicolumn{2}{c}{Nov} & \multicolumn{2}{c|}{Dez} \\
				 \hline
				 % ATIVIDADE             % FV1 FV2 MR1 MR2 AB1 AB2 MA1 MA2 JN1 JN2 JL1 JL2 AG1 AG2 ST1 ST2 OC1 OC2 NV1 NV2 DZ1 DZ2  %
			     %%%%%%%%%%%%%%%%%%%%%%%%%%%%%%%%%%%%%%%%%%%%%%%%%%%%%%%%%%%%%%%%%%%%%%%%%%%%%%%%%%%%%%%%%%%%%%%%%%%%%%%%%%%%%%%%%%%%
				 Escolha do Tema 		   & X & X &   &   &   &   &   &   &   &   &   &   &   &   &   &   &   &   &   &   &   &   \\
				 Escolha do Orientador     & X & X &   &   &   &   &   &   &   &   &   &   &   &   &   &   &   &   &   &   &   &   \\
				 Elicitação de Requisitos  & X & X & X &   &   &   &   &   &   &   &   &   &   &   &   &   &   &   &   &   &   &   \\
				 Escolha das Tecnologias   & X & X & X &   &   &   &   &   &   &   &   &   &   &   &   &   &   &   &   &   &   &   \\
				 Escolha das Ferramentas   & X & X & X &   &   &   &   &   &   &   &   &   &   &   &   &   &   &   &   &   &   &   \\
				 Pesquisa Bibliográfica    &   & X & X & X & X & X & X & X & X & X &   &   &   &   &   &   &   &   &   &   &   &   \\
				 Implementação             &   & X & X & X & X & X & X &   &   &   &   &   &   &   &   &   &   &   &   &   &   &   \\
				 Elaboração Apêndice II    &   &   & X & X & X &   &   &   &   &   &   &   &   &   &   &   &   &   &   &   &   &   \\
				 Documentação              &   &   &   & X & X & X & X & X & X & X & X & X & X & X & X & X & X & X &   &   &   &   \\
				 Fase de Testes            &   &   &   &   &   &   &   & X & X & X & X & X & X & X & X & X &   &   &   &   &   &   \\
				 Aprimoramento \/ Feedback &   &   &   &   &   &   &   &   & X & X & X & X & X & X & X & X & X & X &   &   &   &   \\
				 Apresentação na IFCITEC   &   &   &   &   &   &   &   &   &   &   &   &   &   &   &   & X &   &   &   &   &   &   \\
				 Apresentação Final        &   &   &   &   &   &   &   &   &   &   &   &   &   &   &   &   &   &   & X & X &   &   \\
				 Entrega do Texto Final    &   &   &   &   &   &   &   &   &   &   &   &   &   &   &   &   &   &   &   &   & X &   \\
				 %%%%%%%%%%%%%%%%%%%%%%%%%%%%%%%%%%%%%%%%%%%%%%%%%%%%%%%%%%%%%%%%%%%%%%%%%%%%%%%%%%%%%%%%%%%%%%%%%%%%%%%%%%%%%%%%%%%&
				 \hline
			\end{tabular}
		}
		%%%%%%%%%%%%%%%%%%%%%%%%%%%%%%%%%%%
		%%%        Referências         %%%%
		%%%%%%%%%%%%%%%%%%%%%%%%%%%%%%%%%%%
		\section*{REFERÊNCIAS}
		\addcontentsline{toc}{section}{REFERÊNCIAS}

		\printbibliography[heading=none]

		\newpage
\end{document}
