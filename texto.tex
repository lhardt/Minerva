\documentclass[12pt,a4paper]{article}
\usepackage[utf8]{inputenc}

\usepackage{ifpdf} % configuraç
\usepackage{lipsum} % lorem
\usepackage{scrextend} % addmargin
\usepackage{fontspec} % Arial
\usepackage[portuguese]{babel}
\usepackage{makeidx} % Títulos das tabelas
\usepackage[pagestyles]{titlesec} % titleformat
\usepackage{geometry}
% citestyle authoryear para citações anbt
\usepackage[backend=biber,sorting=nty, style=abnt, citestyle=abnt-numeric]{biblatex}

\addto\captionsportuguese{
	\renewcommand{\contentsname}{ }
	\renewcommand{\contentsname}{ }
	\renewcommand{\listfigurename}{ }
	\renewcommand{\listtablename}{ }
}

\usepackage[breaklinks=true]{hyperref}


\title{MINERVA: UMA SOLUÇÃO INFORMATIZADA PARA O ESCALONAMENTO DE PROFESSORES NAS ESCOLAS BRASILEIRAS}
\ifpdf
	\hypersetup{
		pdftitle    = Minerva,
		pdfauthor   = {Léo Hardt, leom.hardt@gmail.com},
		pdfsubject  = TCC - Léo Hardt,
		pdfcreator  = Léo Hardt,
		pdfproducer = PDFLatex,
		pdfkeywords = {Software, Escola, Grade Horária}
	}
\fi

\geometry{
	a4paper,
	left=30mm,
	right=20mm,
	top=30mm,
	bottom=20mm
}

\pagestyle{myheadings}

\addbibresource{texto.bib}

\setmainfont{Arial}
\setlength{\parindent}{12.5mm}
\renewcommand{\baselinestretch}{1.5}

\titleformat{\section}
  {\normalfont\scshape\bfseries}{\thesection}{1em}{}
\titleformat{\subsection}
	{\normalfont\scshape}{\thesubsection}{1em}{}

\newenvironment{bottompar}{\par\vspace*{\fill}}{\clearpage}

\begin{document}



	%%%%%%%%%%%%%%%%%%%%%%%%%%%%%%%%%%%
	%%%            Capa            %%%%
	%%%%%%%%%%%%%%%%%%%%%%%%%%%%%%%%%%%
	\thispagestyle{empty}

	\begin{center}
		INSTITUTO FEDERAL DE EDUCAÇÃO, CIÊNCIA E TECNOLOGIA DO RIO GRANDE DO SUL - CAMPUS CANOAS \\
		CURSO TÉCNICO DE DESENVOLVIMENTO DE SISTEMAS INTEGRADO AO ENSINO MÉDIO\\
	\end{center}

	\vskip 3cm

	\begin{center}
		LÉO MARCO DE ASSIS HARDT
	\end{center}

	\vskip 5cm

	\begin{center}
		\textbf{MINERVA: UMA SOLUÇÃO INFORMATIZADA PARA O ESCALONAMENTO DE PROFESSORES NAS ESCOLAS BRASILEIRAS}
	\end{center}


	\begin{bottompar}
		\begin{center}
		CANOAS \\
		2020
		\end{center}
	\end{bottompar}

	%%%%%%%%%%%%%%%%%%%%%%%%%%%%%%%%%%%
	%%%       Folha de Rosto       %%%%
	%%%%%%%%%%%%%%%%%%%%%%%%%%%%%%%%%%%
	\thispagestyle{empty}

	\begin{center}
		LÉO MARCO DE ASSIS HARDT
	\end{center}

	\vskip 3cm


	\begin{center}
		\textbf{MINERVA: UMA SOLUÇÃO INFORMATIZADA PARA O ESCALONAMENTO DE PROFESSORES NAS ESCOLAS BRASILEIRAS}
	\end{center}

	\vskip 2cm

	\begin{addmargin}[7.5cm]{0em}

		\setlength{\parindent}{0mm}
		Trabalho de Conclusão de Curso apresentado como requisito parcial para obtenção do grau de Técnico em Desenvolvimento de Sistemas pelo Instituto Federal de Educação, Ciência e Tecnologia do Rio Grande do Sul – Campus Canoas.\\

		Prof. Gustavo Neuberger\\
		Orientador

	\end{addmargin}

	\begin{bottompar}
		\begin{center}
			CANOAS \\
			2020
		\end{center}
	\end{bottompar}


	%%%%%%%%%%%%%%%%%%%%%%%%%%%%%%%%%%%
	%%%       Agradecimentos       %%%%
	%%%%%%%%%%%%%%%%%%%%%%%%%%%%%%%%%%%
	\thispagestyle{empty}
	\section*{AGRADECIMENTOS}

	Gostaria de agradecer, em primeiro lugar, a todos que me apoiaram diretamente. Colegas, amigos, parentes e professores. Em segundo lugar, a todos aqueles que acreditaram num ensino de qualidade e disponibilizaram seu conhecimento ao mundo, mesmo que nunca se conheça o total impacto de tais ações. Nomes dos quais me lembro são Richard Feynmann, Grant Sanderson e Sal Khan. Por fim, gostaria de agradecer aos responsáveis pelo ambiente no qual tive a honra de me formar e no qual gostaria que muitos outros alunos tivessem a mesma oportunidade de evoução que eu tive.

	\newpage


	%%%%%%%%%%%%%%%%%%%%%%%%%%%%%%%%%%%
	%%%     Resumo em Português    %%%%
	%%%%%%%%%%%%%%%%%%%%%%%%%%%%%%%%%%%
	\thispagestyle{empty}
	\section*{RESUMO}

	Um fator decisivo para a eficiência de uma escola é seu cronograma semanal. Além de sua criação ser quase impraticável por uma pessoa, ele pode gerar gastos desnecessários. Sendo assim, programas foram desenvolvidos para a construção da grade horária, cada um levando um sistema educacional em conta. Assim, eles raramente são compatíveis as particularidades de algumas instituições. Além disso, a licença de um software deste ramo tende a ser financeiramente inviável para escolas de baixo orçamento.  O presente trabalho visa, então, a criação de um \textit{software} público de escalonamento de professores que seja capaz de levar em conta as demandas de diversas instituições brasileiras. Uma breve descrição do sistema educacional brasileiro é dada, assim como as necessidades adicionais de \textbf{N} escolas, que representam o grupo como um todo. Para isso, foi utilizada uma abordagem \textbf{abordagem}, de tal forma. \textbf{ -------------- Implementação -------------- }. Após a implantação, o sistema foi testado nessas mesmas \textbf{N} escolas. Por fim, foi disponibilizado um questionário, a fim de obter uma análise diversificada quanto a eficiência e a possíveis melhorias no sistema.

	\begingroup
		\setlength{\parindent}{0mm}
		\textbf{Palavras chave:} \textit{Software}; Escola; Grade Horária.
	\endgroup
	\newpage


	%%%%%%%%%%%%%%%%%%%%%%%%%%%%%%%%%%%
	%%%      Resumo em Inglês      %%%%
	%%%%%%%%%%%%%%%%%%%%%%%%%%%%%%%%%%%
	\thispagestyle{empty}
	\section*{ABSTRACT}

	One crucial factor for a school's efficiency is its weekly timetable. Apart from being impractical to make, it may cause unnecessary spending. With that in mind, various computer programs were designed to create school timetables, each for a specific school system. Thus, they are seldom compatible with the flexibility of some institutions. Furthermore, licenses for this kind of software are often too expensive for low-budget schools. Therefore, the present work aims at the creation of a public timetabling software which takes into account the demands of various brazilian schools. A brief description of the brazilian educational system is given, as well as the necessities of \textbf{N} schools, representing the larger group. For that, \textbf{X approach} was used.  \textbf{ -------------- Implementation -------------- }. After implementation, the system was tested in those same \textbf{N} schools. Finally, a survey was made available for the participants, with the aim of having a diverse analysis of the efficiency and possible improvements to the system.

	\begingroup
		\setlength{\parindent}{0mm}
		\textbf{Keywords:} Software; School; Timetable.
	\endgroup

	\newpage


	%%%%%%%%%%%%%%%%%%%%%%%%%%%%%%%%%%%
	%%%      Resumo em Alemão      %%%%
	%%%%%%%%%%%%%%%%%%%%%%%%%%%%%%%%%%%
	\thispagestyle{empty}
	\section*{ZUSAMMENFASSUNG}

	Ein entscheidende Faktor für die Effizienz einer Schule ist seinen Zeitplan. Zusätzlich zu es

	\lipsum[1]


	\begingroup
		\setlength{\parindent}{0mm}
		\textbf{Schlagworter:} Wort1; Wort2; Wort3.
	\endgroup

	\newpage

	%%%%%%%%%%%%%%%%%%%%%%%%%%%%%%%%%%%
	%%%      Lista de Tabelas      %%%%
	%%%%%%%%%%%%%%%%%%%%%%%%%%%%%%%%%%%
	\thispagestyle{empty}
	\section*{LISTA DE TABELAS}

	\listoftables

	\newpage

	%%%%%%%%%%%%%%%%%%%%%%%%%%%%%%%%%%%
	%%%      Lista de Figuras      %%%%
	%%%%%%%%%%%%%%%%%%%%%%%%%%%%%%%%%%%
	\thispagestyle{empty}
	\section*{LISTA DE FIGURAS}

	\listoffigures

	\newpage


	%%%%%%%%%%%%%%%%%%%%%%%%%%%%%%%%%%%
	%%%      Lista de Siglas       %%%%
	%%%%%%%%%%%%%%%%%%%%%%%%%%%%%%%%%%%
	\thispagestyle{empty}
	\section*{LISTA DE ABREVIATURAS E SIGLAS}

	\begin{tabular}{p{3cm} p{0.6\textwidth}}
	  C & \textit{C Programming Language} \\
	  IFRS & \textit{Instituto Federal de Educação, Ciência e Tecnologia} \\
	  UML & \textit{Unified Modeling Language} \\
	\end{tabular}\\


	\newpage


	%%%%%%%%%%%%%%%%%%%%%%%%%%%%%%%%%%%
	%%%     Lista de Algorítmos    %%%%
	%%%%%%%%%%%%%%%%%%%%%%%%%%%%%%%%%%%
	\thispagestyle{empty}
	\section*{LISTA DE ALGORÍTMOS}

	% \listofalgorithms

	\newpage



	%%%%%%%%%%%%%%%%%%%%%%%%%%%%%%%%%%%
	%%%          Sumário           %%%%
	%%%%%%%%%%%%%%%%%%%%%%%%%%%%%%%%%%%
	\thispagestyle{empty}
	\section*{SUMÁRIO}

	\begingroup
		\let\clearpage\relax
		\vspace{-1cm} % the removed space. Set as appropriate
		\tableofcontents
	\endgroup

	\newpage


	%%%%%%%%%%%%%%%%%%%%%%%%%%%%%%%%%%%
	%%%         Introdução         %%%%
	%%%%%%%%%%%%%%%%%%%%%%%%%%%%%%%%%%%
	\section{INTRODUÇÃO}

		O problema da criação da grade horária escolar é NP-Completo \cite{complexity}. Sendo assim, é um grande entrave no gerenciamento de escolas tanto públicas quanto privadas.

		%%%%%%%%%%%%%%%%%%%%%%%%%%%%%%%%%%%
		%%%      Solução Proposta      %%%%
		%%%%%%%%%%%%%%%%%%%%%%%%%%%%%%%%%%%
		\subsection{SOLUÇÃO PROPOSTA}

			\lipsum[1]


		%%%%%%%%%%%%%%%%%%%%%%%%%%%%%%%%%%%
		%%%      Objetivos Gerais      %%%%
		%%%%%%%%%%%%%%%%%%%%%%%%%%%%%%%%%%%
		\subsection{OBJETIVOS GERAIS}

			\lipsum[1]


		%%%%%%%%%%%%%%%%%%%%%%%%%%%%%%%%%%%
		%%%   Objetivos Específicos    %%%%
		%%%%%%%%%%%%%%%%%%%%%%%%%%%%%%%%%%%
		\subsection{OBJETIVOS ESPECIFICOS}

			\lipsum[1]

	\newpage


	%%%%%%%%%%%%%%%%%%%%%%%%%%%%%%%%%%%
	%%%   Fundamentação Teórica    %%%%
	%%%%%%%%%%%%%%%%%%%%%%%%%%%%%%%%%%%
	\section{FUNDAMENTAÇÃO TEÓRICA}


		Há,


		%%%%%%%%%%%%%%%%%%%%%%%%%%%%%%%%%%%
		%%%   Sistemas Educacionais    %%%%
		%%%%%%%%%%%%%%%%%%%%%%%%%%%%%%%%%%%
		\subsection{SISTEMAS EDUCACIONAIS}

			\lipsum[1]

			%%%%%%%%%%%%%%%%%%%%%%%%%%%%%%%%%%%
			%%%       S. E. Alemão         %%%%
			%%%%%%%%%%%%%%%%%%%%%%%%%%%%%%%%%%%
			\subsubsection{ALEMÃO}

				\lipsum[1]

			%%%%%%%%%%%%%%%%%%%%%%%%%%%%%%%%%%%
			%%%   S. E. Estadounidense     %%%%
			%%%%%%%%%%%%%%%%%%%%%%%%%%%%%%%%%%%
			\subsubsection{ESTADOUNIDENSE}

				\lipsum[1]

			%%%%%%%%%%%%%%%%%%%%%%%%%%%%%%%%%%%
			%%%      S. E. Italiano        %%%%
			%%%%%%%%%%%%%%%%%%%%%%%%%%%%%%%%%%%
			\subsubsection{ITALIANO}

				\lipsum[1]

			%%%%%%%%%%%%%%%%%%%%%%%%%%%%%%%%%%%
			%%%      S. E. Brasileiro      %%%%
			%%%%%%%%%%%%%%%%%%%%%%%%%%%%%%%%%%%
			\subsubsection{BRASILEIRO}

				\lipsum[1]

		\subsection{ALGORÍTMOS UTILIZADOS}

			\lipsum[1]

	\newpage

	%%%%%%%%%%%%%%%%%%%%%%%%%%%%%%%%%%%
	%%%   Trabalhos relacionados   %%%%
	%%%%%%%%%%%%%%%%%%%%%%%%%%%%%%%%%%%
	\section{TRABALHOS RELACIONADOS}

		\lipsum[1]

	\newpage


	%%%%%%%%%%%%%%%%%%%%%%%%%%%%%%%%%%%
	%%%         Conclusão          %%%%
	%%%%%%%%%%%%%%%%%%%%%%%%%%%%%%%%%%%
	\section{CONCLUSÃO}

	\lipsum[1]

		\subsection{TRABALHOS FUTUROS}

			\lipsum[1]

	\newpage


	%%%%%%%%%%%%%%%%%%%%%%%%%%%%%%%%%%%
	%%%        Referências         %%%%
	%%%%%%%%%%%%%%%%%%%%%%%%%%%%%%%%%%%
	\section*{REFERÊNCIAS}
	\addcontentsline{toc}{section}{REFERÊNCIAS}

	\printbibliography[heading=none]

	\newpage

	%%%%%%%%%%%%%%%%%%%%%%%%%%%%%%%%%%%
	%%%         Glossário          %%%%
	%%%%%%%%%%%%%%%%%%%%%%%%%%%%%%%%%%%
	\section*{GLOSSÁRIO}
	\addcontentsline{toc}{section}{GLOSSÁRIO}

	\begin{tabular}{p{3cm} p{0.6\textwidth}}
	  Software & \textit{Programa de Computador} [referenciar] \\
	  Dolor & \textit{Instituto Federal de Educação, Ciência e Tecnologia} \\
	  Sit Amet & \textit{Unified Modeling Language} \\
	\end{tabular}\\


	\newpage

	%%%%%%%%%%%%%%%%%%%%%%%%%%%%%%%%%%%
	%%%          Apêndice          %%%%
	%%%%%%%%%%%%%%%%%%%%%%%%%%%%%%%%%%%
	\section*{APÊNDICE}
	\addcontentsline{toc}{section}{APÊNDICE}

	\lipsum[1]

	\newpage

\end{document}
