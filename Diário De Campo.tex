
\documentclass{article}
\usepackage[a4paper, top=3cm, left=3cm, right=2cm, bottom=2cm]{geometry}
\begin{document}

\vspace{0.3cm}\hrule\vspace{0.3cm}
\large{\bf{Sobre o Problema do Escalonamento de Professores}} \par
Léo H.
\vspace{0.4cm}\hrule\vspace{0.4cm}

\section{Condições para a aplicação no Ensino Fundamental}

\subsection{Particularidades do Ensino Fundamental Brasileiro}

1. A semana (um ciclo da agenda) será dividida em 5 dias, subdivididos em 5 períodos, sendo que entre o terceiro e o quarto há um intervalo.

2. É sempre pré-decidido quais períodos uma turma


\subsection{"Hard Constraints"}

1. Não há um mesmo período em que duas turmas tem aula com um mesmo professor.

2. Não há um mesmo período em que dois professores dão aula para uma mesma turma.

3. O número de vezes que um professor se encontra com uma turma é dado.



\section{Desenvolvimento Diário}

\subsubsection{30 Oct 2019}

Me parece que trabalhar na perspectiva do professor parece bem prudente. Ao invés de desenvolver na ordem Teacher - TeacherQuantity - Class, acredito que o contrário seja mais prudente. Como os horarios dos professores são infinitamente mais limitados, há muito maior possibilidade de propagação de constraint.

\subsubsection{31 Oct 2019}

Talvez trabalhar com os dois aspectos ao mesmo tempo seja mais proveitoso. Só ainda não sei como aproveitar isso.

\subsubsection{21 Nov 2019}

\par A ideia da Command Line Interface foi maaais ou meeenos colocada no lugar. O ponto é que podemos escolher duas coisas:

1. Fazer constraint propagation e sugestões pra quem tiver montando. E dai uma interface bem completinha ajudaria na solução do problema.

2. Tentar deduzir por heurística de uma vez só uma agenda. Tem menos chance de sucesso, no sentido que demora mais e dá menos controle para o usuário. Além disso, é pior para fazer um fact-check e descobrir se está boa a agenda.

\par O que eu vejo é que a primeira opção é melhor, mas menos comum na indústria. É só que o esforço teria que ser maior na parte
de interface gráfica. Então se tivermos uma CLI robusta o suficiente, podemos ter uma GUI que só se utiliza de CLI.

\par Logo, para tudo funcionar bem, precisamos de funcionalidades como:

-     ADD  (professor, aluno, turma, disciplina, cronograma);

- ```SELECT (professor, aluno, turma, disciplina, cronograma);```

- ```DELETE (professor, aluno, turma, disciplina, cronograma);```

- ```UNDO (bem importante para desfazer merdas);```

- ```REDUCE cronograma;```




\end{document}
